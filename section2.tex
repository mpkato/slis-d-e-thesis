\chapter{Related Work}
\label{sq:related_work}
In this section, we provide a literature review in the fields related to this work, which are fashion recommendation and interactive recommender systems.

% よく引用されているメジャーな論文は網羅的に引用して,簡単に内容を説明してインタラクティブなファッション推薦はない,ということを主張する
\section{Fashion Recommendation}
Fashion, a form of self-expression and autonomy at a specific time or location, has been a large part of daily life. Generally, clothing is the most crucial part of fashion. People are constantly debating which clothes to wear or purchase. The fashion recommendation task seeks to recommend appropriate fashion items or outfits, as well as a series of fashion items, based on given information such as the user's purchase history and item features. 

Kang et al. proposed an support vector machine (SVM) and collaborative-filtering-based fashion recommender systems to improve customer satisfaction~\cite{kang2007svm}. 
Liu et al. introduced an occasion-oriented clothing recommender system that relies on a latent SVM framework with attributes and categories~\cite{liu2012hi}.
With the massive success of online apparel sales in the last decade and a large amount of user data, many fashion recommender systems have been proposed in recent years~\cite{wang2015joint, hu2015collaborative, kang2017visually, li2017mining}. 
Li et al.~\cite {li2017mining} presented a recurrent neural network (RNN) model with fused features of the item to predict its popularity. 
Through large-scale experiments, they confirmed that the proposed method is efficient.

In contrast to fashion item recommendation, the outfit recommendation task requires an extra ability to evaluate the compatibility of items~\cite{chen2019pog, li2020hierarchical, lin2020outfitnet, verma2020addressing}. 
The performances of fashion outfit recommender systems are evaluated in the following two tasks:
\begin{description}
    \item[Fill in the blank] This task is to choose from multiple items to complete an outfit. For instance, given four items,
     as well as three candidate items, the task is to choose one of the candidate items that best fits as the fifth item to complete an outfit consisting of five items.
    \item[Compatibility prediction] This task is to predict the compatibility of the given outfit. The compatibility of a fashion outfit measures the suitability of the items to the style of the overall outfit. In most cases, the compatibility of the outfit indicates the affinity between each pair of items.
\end{description}

Notably, fashion is constantly changing; the opinion toward fashion is unstable. Many studies on fashion recommendation suggest that contextual information need to be counted in recommendation~\cite{shen2007gonna, tamhane2017modeling}.
When the context is not informative enough to capture rapidly changing users' preferences, 
users' feedback is necessary to understand the current mood of users.
However, there are no studies that actively elicit users' preferences for fashion recommendation. 


\section{Interactive Recommender Systems}

Recommender systems have been widely used in various services for increasing commercial revenue.
However, there are two problems in the existing recommendation systems:
(1) users' experiences are monotonous: users can only passively receive recommended items~\cite{sinha2002role, herlocker2000explaining}, and
(2) systems have limited knowledge about users for effective recommendation~\cite{jugovac2017interacting}.
A possible solution to alleviate these two problems is 
to allow users to interact with the recommender systems.
Interactions enable users to feel more involved and intervene in recommendation results to a certain extent, rather than passively receiving results~\cite{zhu2019query}.
Moreover, the system can better understand the users' preferences through interactions, and, accordingly, improve the recommendation results.

Christakopoulou et al.~\cite{christakopoulou2018q} proposed a single-round interactive framework with three modules: question generation, user feedback, and item recommendation.
The user needs to answer the question generated by the system, 
and the system predicts relevant items based on the user's feedback.

The effectiveness of interactive recommender systems has been shown in many fields~\cite{adomavicius2011context, chen2012cofeel, bogdanov2013semantic, bakalov2013approach}.
While the interactive recommendation is thought to be a superior solution in some applications
where users are willing to provide feedback,
the evaluation of interactive recommendation systems is challenging 
as it often requires real recommendation services.
Another evaluation methodology is a simulation 
in which users behaviors are simulated based on the observation in real services.
While the simulation may not fully reflect real users' behaviors,
the evaluation is reproducible and easily repeatable. 
Thus, simulation-based evaluation can be useful for initial development of interactive systems.
We took the latter approach in this study and developed 
the dataset for evaluating interactive fashion outfit recommender systems through simulations.